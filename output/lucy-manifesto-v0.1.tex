\documentclass[11pt]{article}
\usepackage[margin=1in]{geometry}
\usepackage[T1]{fontenc}
\usepackage[utf8]{inputenc}
\usepackage{lmodern}
\usepackage{hyperref}
\usepackage{booktabs}
\usepackage{graphicx}

\title{Lucy Manifesto v0.1 Gate Package}
\author{The Lucy Manifesto Project}
\date{\today}

\begin{document}
\maketitle

\section{Intent}
This document consolidates the v0.1 gate logic and artifacts for the Lucy revitalization effort.
It is based on the cleaned source material and translates author intent into an engineering baseline.

\section{Core Position}
\begin{itemize}
  \item Lucy is a developmental intelligence platform, not a scripted robot product.
  \item Intelligence must emerge from embodied sensorimotor loops.
  \item Distributed, low-power compute is the preferred implementation path.
\end{itemize}

\section{Manifesto v0.1 Summary}
\begin{itemize}
  \item Preserve body-first intelligence and biologically meaningful sensing/actuation.
  \item Move from PIC-era distributed control to GA144 manycore distributed runtime.
  \item Prioritize calibration, reflex loops, and stable closed-loop behavior before symbolic features.
  \item Keep scope to torso/head/arms for v0.1; defer full biped and production design.
\end{itemize}

\section{Target Outcome}
The v0.1 documents and gate criteria are oriented toward a single idealized outcome: \textbf{Lucy Mk-II}.
This figure is the integration target that aligns architecture, requirements, and tests.

\begin{center}
\includegraphics[width=0.68\linewidth]{../figures/Lucy-Mk-II.png}\\
\textit{Lucy Mk-II prototype concept}
\end{center}

\section{Requirements Baseline}
\subsection*{Functional}
\begin{itemize}
  \item R1: Execute distributed real-time control/perception on GA144 fabric.
  \item R2: Deterministic message passing between sensor, motor, and integration modules.
  \item R3: Calibration and closed-loop control for at least 3--4 controllable DoF.
  \item R4: Reduced, contrast-focused vision preprocessing output.
  \item R5: Banded audio features and directional cue output from dual microphones.
  \item R6: Virtual-muscle style command interface at high level.
  \item R7: Timestamped telemetry for sensor, motor, and safety events.
\end{itemize}

\subsection*{Safety}
\begin{itemize}
  \item S1: Joint range enforcement.
  \item S2: Current/thermal guard enforcement where available.
  \item S3: Watchdog-triggered safe-mode behavior on timeout/comm failure.
  \item S4: Repeatable startup calibration with fail-safe abort on invalid calibration.
\end{itemize}

\subsection*{Non-Functional}
\begin{itemize}
  \item N1: Stable control-loop timing sufficient to avoid baseline oscillation.
  \item N2: Graceful degradation under non-critical module failure.
  \item N3: Externalized configuration/calibration parameters.
  \item N4: Full requirement-to-test traceability.
\end{itemize}

\section{GA144 Architecture (v0.1)}
\begin{itemize}
  \item M1 Sensor Ingest
  \item M2 Perception Preprocess
  \item M3 Integration/Reflex
  \item M4 Motor Interface
  \item M5 Safety Supervisor
  \item M6 Telemetry
\end{itemize}

\subsection*{Messaging Model}
\begin{itemize}
  \item Deterministic message passing with fixed schemas.
  \item Control-relevant messages include timestamp, source ID, sequence, and payload version.
\end{itemize}

\subsection*{Startup and Safety}
\begin{itemize}
  \item Startup phases: hardware check, sensor sanity, calibration, closed-loop readiness.
  \item Failure classes: recoverable transient, degraded operation, fatal/safe-stop.
\end{itemize}

\section{Test Strategy (v0.1)}
\subsection*{Validation Targets}
\begin{tabular}{@{}p{0.18\linewidth}p{0.24\linewidth}p{0.50\linewidth}@{}}
\toprule
Target & Required Tests & Pass Condition \\
\midrule
VT1 & IT-01, ST-01 & Stable 3--4 DoF closed-loop baseline control without unstable oscillation. \\
VT2 & IT-02, ST-02 & Consistent orient-to-sound response from directional cues. \\
VT3 & IT-03, ST-03 & Consistent fixation/tracking behavior using reduced vision path. \\
VT4 & IT-04, ST-04 & Repeatable startup calibration with comparable calibration metrics. \\
\bottomrule
\end{tabular}

\subsection*{Exit Criteria}
\begin{itemize}
  \item VT1--VT4 pass with reproducible evidence.
  \item No unresolved critical safety defects.
  \item Requirement traceability matrix complete and current.
\end{itemize}

\section{Gate Artifacts}
\begin{itemize}
  \item \texttt{docs/manifesto-v0.1.md}
  \item \texttt{docs/requirements.md}
  \item \texttt{docs/architecture-ga144.md}
  \item \texttt{docs/test-strategy.md}
\end{itemize}

\end{document}
